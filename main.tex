%%%%%%%%%%%%%%%%%% USAGE INSTRUCTIONS %%%%%%%%%%%%%%%%%%
% - Compile using LuaLaTeX and biber, unless there is a particular reason not to. Do not use the older LaTex/PDFLaTeX or BibTeX. (The fonts won't work correctly.)
% - Font and the report 'year' must be specified when all \documentclass or the template won't work correctly. (There's no error checking/default cases!)
% - Options for fonts are: calibri, times, palatino, garamond, arial, tahoma, verdana, trebuchet. Use times unless there is a particular reason not to. Calibri would be the next choice. 
% - Options for 'year' are: first, second, thesis. 
% - For best performance save images/graphics as PDF files, not as png/jpg/eps. This makes no difference to how images are inserted using \includegraphics.
% - As many further packages as wanted can be loaded. Below are just an example set. Note that template itself loads a number of packages, including hyperref.
% - References are handed using biblatex.
% - Don't need to include a \uomdeclarations unless this is a thesis
% - Link to the presentation of theses policy: http://www.regulations.manchester.ac.uk/pgr-presentation-theses/



%%%%%%%%%%%%%%%%%% DOCUMENT SETUP %%%%%%%%%%%%%%%%%%
\documentclass[times,thesis]{uom_thesis_casson} % Font should be calibri, can use times if necessary. Year can be: first, second or thesis.



%%%%%%%%%%%%%%%%%% PACKAGES AND COMMANDS %%%%%%%%%%%%%%%%%%

% Packages
\usepackage{graphicx,psfrag,color} % for postscript graphics files
  \graphicspath{ {./images/} }
\usepackage{amsmath}               % assumes amsmath package installed
  \allowdisplaybreaks[1]           % allow eqnarrays to break across pages
\usepackage{amssymb}               % assumes amsmath package installed 
\usepackage{url}                   % format hyperlinks correctly
\usepackage{rotating}              % allow portrait figures and tables
\usepackage{multirow}              % allows merging of rows in tables
\usepackage{lscape}                % allows pages to be typeset in landscape mode
\usepackage{tabularx}              % allows fixed width tables
\usepackage{verbatim}              % enhanced version of built-in verbatim environment
\usepackage{footnote}              % allows more control over footnote environments
\usepackage{float}                 % allows H option on floats to force here placement
\usepackage{booktabs}              % improve table line spacing
\usepackage{lipsum}                % for adding dummy text here
\usepackage{subcaption}            % for multiple sub-figures in a single float
% Add your packages here

% Optional: for adding alt-text to images:
\usepackage{pdfcomment}            % for alt text for accessibility


% Custom commands
\newcommand{\degree}{\ensuremath{^\circ}}
\newcommand{\sus}[1]{$^{\mbox{\scriptsize #1}}$} % superscript in text (e.g. 1st)
\newcommand{\sub}[1]{$_{\mbox{\scriptsize #1}}$} % subscript in text
\newcommand{\chap}[1]{Chapter~\ref{#1}}
\newcommand{\sect}[1]{Section~\ref{#1}}
\newcommand{\fig}[1]{Fig.~\ref{#1}}
\newcommand{\tab}[1]{Table~\ref{#1}}
\newcommand{\equ}[1]{(\ref{#1})}
\newcommand{\appx}[1]{Appendix~\ref{#1}}
% Add your commands here



%%%%%%%%%%%%%%%%%% REFERENCES SETUP %%%%%%%%%%%%%%%%%%
\usepackage[backend=biber,style=ieee,bibencoding=utf8,hyperref=auto]{biblatex}
  \addbibresource{references.bib}
  % Add more .bib files here if wanted

% Bug fix which can't be applied in the .cls file
\renewcommand*{\bibfont}{\large} 


%%%%%%%%%%%%%%%%%% START DOCUMENT %%%%%%%%%%%%%%%%%%
\begin{document}



%%%%%%%%%%%%%%%%%% COVID-19 impact statement %%%%%%%%%%%%%%%%%%
\begin{uomcovid} % policy asks for statement to be before the title page
  % Policy asks for this to be removed from the final thesis post-examination, which messes up the page numbers somewhat. Done here so that the final post-examination thesis is correct. In the pre-examination thesis the numbers displayed on the page will be one lower than the number displayed by the PDF reader
  This is COVID-19 impact text.
  
  \lipsum[1-3] % generate dummy text for here
\end{uomcovid}% \clearpage is added automatically



%%%%%%%%%%%%%%%%%% TITLE PAGE %%%%%%%%%%%%%%%%%%
\title{A data reduction algorithm incorporating a low power continuous wavelet transform for use in wearable electroencephalography systems}
\author{Alexander J.\ Casson}
\faculty{Science and Engineering}                  % "Faculty of" is added automatically
\department{Electrical and Electronic Engineering} % "Department of" is added automatically
\submitdate{2021}                                  % regulations ask only for the year, not month
\newcommand{\wordcount}{X}		                   % manually replace 'X' with the word count of document before submitting - only displayed in for a thesis
\maketitle



%%%%%%%%%%%%%%%%%% LISTS OF CONTENT %%%%%%%%%%%%%%%%%%
\uomtoc % probably don't need all of these unless final thesis
\uomlof
\uomlot
\begin{uomlop} % list of publications.
  % Can use biblatex to \printbibliography[heading=none] to populate automatically, but probably easier to just type in
  Publications go here.
\end{uomlop}
\begin{uomterms}
  Enter terms and abbreviations as table or similar
  % Add list of terms and abbreviations by hand if wanted. Is no formal requirement to have one
\end{uomterms}



%%%%%%%%%%%%%%%%%% ABSTRACT %%%%%%%%%%%%%%%%%%
\begin{abstract} % put abstract here. Limit is 1 page.
  This is abstract text. 
  
  \lipsum[1-2]
\end{abstract}%
\clearpage



%%%%%%%%%%%%%%%%%% LAY ABSTRACT %%%%%%%%%%%%%%%%%%
\begin{uomlay} % put lay abstract here. Limit is 1 page. Not compulsory
  This is lay abstract text. 
  
  \lipsum[1-2]
\end{uomlay}



%%%%%%%%%%%%%%%%%% DECLARATIONS %%%%%%%%%%%%%%%%%%
\uomdeclarations % Don't need unless final thesis



%%%%%%%%%%%%%%%%%% ACKNOWLEDGEMENTS %%%%%%%%%%%%%%%%%%
\begin{uomacknowledgements} % probably don't need unless final thesis
Acknowledgements go here.
\end{uomacknowledgements}



%%%%%%%%%%%%%%%%%% CHAPTER 1 %%%%%%%%%%%%%%%%%%
\chapter{Introduction} % probably better to use \input{} for each chapter to draw content from other files rather than having everything in one big file
  \lipsum[1-5] % generate dummy text for here

\chapter{Literature review}

  \section{Introduction}
  \lipsum[1] 
  
  \section{Content}
  \lipsum[1-2] \cite{ref:jCAS09,ref:jCAS09a,ref:jCAS10} \lipsum[3-5]
  \begin{table}
    \centering
    \caption{Example table.}
    \label{table:example_table}
    \begin{tabular}{ccccc}
      \toprule
      \multirow{2}{*}{Participant} & \multicolumn{2}{c}{Number / \%} & \multicolumn{2}{c}{Duration / \%} \\
                                   & Prime dresses & Non-prime dresses & Prime dresses & Non-prime dresses \\
	  \toprule
      1  & 33.33 & 33.91 & 20.83 & 18.42 \\
      2  & 13.04 & 17.50 & 04.93 & 07.62 \\
      3  & 22.73 & 20.10 & 13.00 & 08.20 \\
      4  & 31.34 & 21.88 & 10.57 & 11.09 \\
      5  & 08.47 & 19.32 & 03.04 & 09.73 \\
      \hline
      Mean & 16.4 & 16.5 & 07.8 & 07.5 \\
      Standard deviation & 09.7 & 06.6 & 05.4 & 03.3 \\
      \bottomrule
    \end{tabular}
  \end{table}  
  
  This is an example equation
  \begin{equation}
      a^{2} + b^{2} = c^{2}
  \end{equation}
  
  \section{Summary}
  \lipsum[6]
  
  
  
%%%%%%%%%%%%%%%%%% CHAPTER 2 %%%%%%%%%%%%%%%%%%
\chapter{Really good work}

  \section{Introduction}
  \lipsum[1] 
  
  \section{Content}
    \subsection{Introduction}
	\lipsum[1]
	
	\subsection{Detail}
    \lipsum[1-2] \cite{ref:jCAS10,ref:jCAS09,ref:jCAS09a} \lipsum[3-5]
	\begin{figure}
      \centering
      \includegraphics[width=0.3\textwidth]{uom_logo.pdf}
      \caption{Example figure.}
      \label{fig:uom_logo}
    \end{figure} 
    
	\subsection{More detail}
	\lipsum[1-2] \cite{ref:jCAS10,ref:jCAS09,ref:jCAS09a} \lipsum[3-5]
	
	\subsection{Summary}
	\lipsum[1]
  
  \section{Summary}
  \lipsum[6]
 



%%%%%%%%%%%%%%%%%% CONCLUSIONS %%%%%%%%%%%%%%%%%%
\chapter{Conclusions}



%%%%%%%%%%%%%%%%%% REFERENCES %%%%%%%%%%%%%%%%%%
\printbibliography[title={References},heading=bibintoc] % a single list of references for the whole thesis



%%%%%%%%%%%%%%%%%% APPENDICIES %%%%%%%%%%%%%%%%%%
\begin{uomappendix} 
  \chapter{First appendix}
    \section{Section in Appendix}
    \lipsum[1-6]
\end{uomappendix}


%%%%%%%%%%%%%%%%%% END MATTER %%%%%%%%%%%%%%%%%%
\end{document}